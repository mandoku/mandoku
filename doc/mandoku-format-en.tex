% Created 2014-03-12 Wed 12:37
\documentclass[11pt]{article}
\usepackage{fontspec}
\usepackage{xunicode}
\usepackage{xeCJK}
\usepackage{xltxtra}
\usepackage{graphicx}
\usepackage{longtable}
\usepackage{float}
\usepackage{wrapfig}
\usepackage{soul}
\usepackage{textcomp}
\usepackage{hyperref}
\usepackage[backend=biber,natbib=true,style=authoryear,babel=hyphen,isbn=false,doi=false,url=false]{biblatex}
\setCJKmainfont[Scale=0.85]{PMingLiU}
\tolerance=1000
\author{Christian Wittern}
\date{\today}
\title{The \texttt{Mandoku} text format}
\hypersetup{
  pdfkeywords={},
  pdfsubject={},
  pdfcreator={Emacs 24.3.1 (Org mode 8.2.5a)}}
\begin{document}

\maketitle
Conventions for the text format used by \texttt{Mandoku} are made up of three parts:
\begin{itemize}
\item Emacs org-mode format
\item convention about how texts are organized into files
\item Additional syntactical conventions singular to \texttt{Mandoku} within the files
\end{itemize}

\section{The text format is based on Emacs org-mode}
\label{sec-1}

The text editor Emacs has the concept of "modes", which allows for
specialized modes of editing depending on the type of text to be
edited. This works for various programming languages and helps the
programmer to avoid mistakes by giving graphical hints through
formatting and colors and in general support the task of writing in
this language.  

Emacs org-mode (see \href{http://orgmode.org/}{Org-Mode: Your Life in Plain Text}[orgmode.org]) builds on
this concept, but the target is not a specific programming language
but plain text with some outlining, formatting and hyperlinking
conventions.  This makes it a lightweight text format that covers
some middle ground between absolutely complete plain text (if such a
thing exists) and complex markup languages like XML.  Since it even
supports footnotes and is easily converted to HTML, Latex and
OpenOffice format, it is useable even for research articles.

To make this possible, org-mode introduces a few special markup
constructs. Some of the assumptions of this plain text according to
org-mode are as follows:

\begin{itemize}
\item Headings of all kinds are expressed by placing stars "* " in the
leftmost column of the text. The number of stars indicates the
level of the heading in the overall outline of the document. The
stars have to be followed by a space in order to be recognized as
a part of the outline. Structurally, a heading defines a
subsection.
\end{itemize}
\begin{verbatim}
* Outline level one レベル1の見出し
** Outline level two レベル2の見出し
*** Outline level three レベル3の見出し
\end{verbatim}
\begin{itemize}
\item following a heading a socalled "property drawer" may appear; this
is a container for additional information associated with the
subsection.
\end{itemize}
\begin{verbatim}
* Introduction
 :PROPERTIES:
 :CUSTOM_ID: intro
 :END:
\end{verbatim}
\begin{itemize}
\item Headings can also have "tags" attached to them, which marks them
for example for various analytical purposes, they appear on the
headline towards the right margin.
\end{itemize}
\begin{verbatim}
* Outline level one レベル1の見出し          :important:
\end{verbatim}
\begin{itemize}
\item Paragraphs are indicated by at least one empty line.
\item Distinctive parts of the text, for example quotations or parts in
verse can be set apart using a specific section marker:
\end{itemize}
\begin{verbatim}
即於佛前以偈頌曰。
#+BEGIN_VERSE
目淨脩廣如青蓮
心淨已度諸禪定
久積淨業稱無量
導眾以寂故稽首
#+END_VERSE
\end{verbatim}
\begin{itemize}
\item Lines beginning with a pound sign '\#' are treated specially, not as part of the text:
\begin{itemize}
\item if followed by a "+" they introduce special keywords or syntax
used for org-mode's own purposes, for example \texttt{\#+TITLE},
\texttt{\#+AUTHOR} or \texttt{\#+DATE}.
\item the pound sign followed by a space character "\# " introduces a
comment which is not considered part of the document.
\end{itemize}
\item Links to other files target.txt or places within the current file
are constructed like this: \texttt{[[target.txt][link to target.txt] ]}
\end{itemize}

\section{Additional conventions for \texttt{Mandoku}}
\label{sec-2}


\subsection{Text identifiers}
\label{sec-2-1}
As of September 2013, Mandoku switched to a new type of text
identifiers. This identifier, such as \texttt{ZB6i0076} is composed of three parts:
\begin{itemize}
\item "ZB", which is the repository identifier -- this indicates the origin of the text
\item "6i", which indicates in which subcollection the text is found,
in this case "經集部類", which is part of the top-level grouping
"6 佛部".
\item "0076" which is the serial number of the text in the collection.
\end{itemize}

Most texts of the part "6 佛部" of the ZB repository do derive from
the CBETA collection of Buddhist texts, which has a different set
of text identifiers, in this case, the text in question has the
title "維摩詰所說經" and the CBETA identifier \texttt{T14n0475}, this was
also used in \texttt{Mandoku} in previous versions.
\subsection{About texts, versions and editions}
\label{sec-2-2}

In \texttt{Mandoku} every text has a text identifier, for example
\texttt{ZB6i0076=}, which uniquely identifies the text within a larger
collection and makes it possible to refer to it unambiguoqusly.
This identifier is usually also the name of the folder, where all
files that make up a text are stored. The text is usually split up
so that one file has the content of one scroll, and they are
numbered in sequence.  An additional file may give the table of
contents with links to the appropriate location in the files.

\begin{verbatim}
ZB6i0076_000.txt
ZB6i0076_001.txt
ZB6i0076_002.txt
ZB6i0076_003.txt
ZB6i0076.org
\end{verbatim}


The source of the text has to be given (at the beginning of the
file, this is recorded with the property keyword "WITNESS") and
there might also be a base edition (keyword "BASEEDITION"). The
base edition provides the navigation grid for all editions of a
text.

Several editions of a text might be recorded; this allows the
documentation of textual witnesses for a text and provides the
ground for critical editions of the text. Within \texttt{Mandoku},
different editions (witnesses) of a text are stored as "branches"
in a version control system.

In order to provide a common reference system common to all
editions of a text, page numbers and locations of the line break of
the base edition are recorded in all files. Editions that have a
different layout may additionally record the page numbers according
to the source edition.

\subsection{Syntactic conventions}
\label{sec-2-3}

\subsubsection{Base edition and text witness}
\label{sec-2-3-1}
In \texttt{Mandoku} some import information about the source and
edition of a text can be given at the beginning of a file in machine
readable form, for example:

\begin{verbatim}
#+PROPERTY: BASEEDITION T
#+PROPERTY: ID ZB6i0076
#+PROPERTY: CBETA_ID T14n0475
#+PROPERTY: WITNESS 【CBETA】
#+PROPERTY: JUAN 1
\end{verbatim}

Here the base edition is identified as "T" and the witness (the
edition in this file) is given as "【CBETA】".  In addition, the
identification number of the text is recorded and the number of the
juan. As mentioned above, the text identifiers have been changed
recently, therefore the previously used CBETA\_ID is also given for
reference.
\subsubsection{Page numbers}
\label{sec-2-3-2}
The page number is given in the following form:
There are three parts to this number, separated by the "\_"
character, surrounded by the page break indicator "":
\begin{itemize}
\item "ZB6i0076" is the identification number of the text
\item "T" is the identifier of the edition
\item "0537a" is the page number ("a" stands for the upper part of the
text or the first half of a page in woodblock prints that are
separated in two halves) At the moment \texttt{Mandoku} requires the page
number to end in a letter.
\end{itemize}

The page break indicator ": on a line by themselves and are ended with :END: again on
  a line by themselves. In interactive use, the content of drawers
  can be hidden and made to appear only when needed.
% Emacs 24.3.1 (Org mode 8.2.5a)
\end{document}
